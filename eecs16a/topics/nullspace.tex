\section{Null Space}

\begin{frame}{Null Space}
    \begin{itemize}
        \item \textbf{Definition}: The \textbf{null space} of a matrix (transformation) is the set of all solutions to the homogeneous equation $A\vec{x} = \vec{0}$.
        \item It is a \textit{subspace of $\mathbb{R}^n$}.
        \item Solving a null space:
        \begin{enumerate}
            \item Reduce to \textbf{reduced row echelon form}.
            \item Find solution to the system of equations.
            \item Represent the solutions in \textit{matrix form}.
        \end{enumerate}
    \end{itemize}
\end{frame}

\begin{frame}{Practice: Null Space}
    Find the \textbf{null space} of $A$:
    \begin{align*}
        A = \begin{bmatrix}
            -3 & 6 & -1 & 1 & -7 \\
            1 & -2 & 2 & 3 & -1 \\
            2 & -4 & 5 & 8 & 2
        \end{bmatrix}
    \end{align*}
\end{frame}

\begin{frame}{Practice: Null Space [Solution]}
    First, \textbf{row reduce}.
    \begin{align*}
        A = \begin{bmatrix}
            -3 & 6 & -1 & 1 & -7 \\
            1 & -2 & 2 & 3 & -1 \\
            2 & -4 & 5 & 8 & 2
        \end{bmatrix} \longrightarrow
        \begin{bmatrix}
            1 & -2 & 0 & -1 & 0 \\
            0 & 0 & 1 & 2 & 0 \\
            0 & 0 & 0 & 0 & 1
        \end{bmatrix}
    \end{align*}
\end{frame}

\begin{frame}{Practice: Null Space [Solution]}
    \begin{itemize}
        \item We saw that in the row-reduced matrix, there were \textbf{3 pivot columns}.  
        \item Additionally, we know that there are 5 total “variables” 
        \item Thus, we can say that there are \textbf{2 free variables}, and \textit{obtain a basis for our null space in terms of these free variables}!
        \item The \textbf{pivot columns} occur at $x_1, x_3$, and $x_5$, so we can set $x_2 = r$ and $x_4 = s$
        \item Let's find our basis in terms of $r, s$!
    \end{itemize}
\end{frame}

\begin{frame}{Practice: Null Space [Solution]}
    Let's find our basis in terms of $r, s$!
    \begin{align*}
        \begin{bmatrix}
            1 & -2 & 0 & -1 & 0 \\
            0 & 0 & 1 & 2 & 0 \\
            0 & 0 & 0 & 0 & 1
        \end{bmatrix} \\
        x_1 = 2r + s, \,\,\, x_2 = r \\
        x_3 = -2s, \,\,\, x_4 = s, \,\,\, x_5 = 0
    \end{align*}
    In matrix form:
    \begin{align*}
        \vec{x} = r \begin{bmatrix}
            2 \\ 1 \\ 0 \\ 0 \\ 0
        \end{bmatrix} + s \begin{bmatrix}
            1 \\ 0 \\ -2 \\ 1 \\ 0
        \end{bmatrix}
    \end{align*}

\end{frame}