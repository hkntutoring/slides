
	\section{Principle Component Analysis}

	\begin{frame}{PCA}
	PCA is a linear dimensionality reduction tool. Given data $\vec{x}_i \in \mathbb{R}^d$, we can create a mapping $T : \mathbb{R}^d \rightarrow \mathbb{R}^{d'}, d' < d$ such that the variance in the dataset is still captured
	\end{frame}

	\begin{frame}{PCA --- Computation}
		\begin{enumerate}[<+->]
			\item Store data row-major in $A \in \mathbb{R}^{n \times d}$
			\item De-mean $A$
			\item Take SVD: $A = U \Sigma V^T$
			\item Create $V_{d'} \in \mathbb{R}^{n \times d'}$ from vectors of $V$ corresponding to $d'$ greatest singular values
			\item To project data into the representative subspace: $T(\vec{x}) := V_{d'}^T \vec{x}$
		\end{enumerate}
	\end{frame}

	\begin{frame}{PCA: Computation}
	The mapping $T$ can then be expressed as
	\[ T(\vec{x}) = V_k^T \vec{x} \]
	If we apply this transformation onto the entire dataset (which has \emph{row vectors}), we can say
	\[ T(A) = B = V_k^T A\]
	where $B \in \mathbb{R}^{n \times k}$.
	\end{frame}
	\begin{frame}{PCA: Computation}
	If we were to show the projected vectors in the original space, we can multiply back with the projection vectors
	\[ A' = B V_k^T \]
	\end{frame}

