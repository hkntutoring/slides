\section{Diagonalization}

\begin{frame}{Diagonalization}
	    \begin{itemize}
	        \item The idea is that we want to change into a basis in which the system $A\vec{x} = \vec{y}$ is represented by a diagonal matrix. So how do we find such basis?
	        
	        \item Remember that for all eigenvalue-eigenvector pairs we have: $A\vec{v} = \lambda\vec{v}$
	        
	        \item Let’s use our eigenvectors as our basis. Doing so we obtain:
	        
	        $$\vec{x} = V\tilde{\vec{x}}, \vec{y} = V\tilde{\vec{y}}, \text{and} \Lambda\tilde{\vec{x}} = \tilde{\vec{y}}$$
	        
	        Where the upper-case lambda represents the diagonal matrix with eigenvalues on the diagonals.
	        
	        \item Transforming back to the standard basis, we get:
            
            $$A = V\Lambda{V^{-1}}$$
	    \end{itemize}
	\end{frame}
	
	\begin{frame}{Diagonalization Cont.}
	    \begin{itemize}
	        \item Let’s analyze this a bit further, why is it important to be able to do this?
            
            \item We see that diagonalizing the matrix makes the system much easier to solve, why? 
            
            \item Also, we see that there is a “home state” for every system of linearly independent equations, i.e. the space in which the system’s components are uncoupled.
	    \end{itemize}
	\end{frame}
	
