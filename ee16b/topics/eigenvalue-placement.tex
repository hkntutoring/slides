
	\section{Eigenvalue Placement}

	\begin{frame}
	\frametitle{Eigenvalue Placement} 
	\centering\includegraphics[width = 10cm]{./images/eigenvalue.jpg}
\end{frame}

\begin{frame}
\frametitle{Why?} 
\begin{itemize}
\item Recall that we are always interested in determining if a given system is BIBO (bounded input bounded output) stable. 
\item More precisely, if we have a system described by $\vec{x}(t+1) = A\vec{x}(t) + Bu(t) + \vec{\omega}(t)$ we would like the eigenvalues of $A \in \mathbb R^{n \times n}$, to satisfy the following property : $|\lambda_i| < 1$. 
\item So what if we have a $\lambda$ that does not satisfy this property?
\item This is where eigenvalue placement comes into play!
\item Assuming the system is controllable, we will use closed loop controls to change the eigenvalues such that they satisfy this property. 
\end{itemize}
\end{frame}

\begin{frame}
\frametitle{How?}
\begin{itemize}
\item Assume e.g. a DT system. Input: $u[t]$ If the system is controllable then we can use feedback, which means that we can let the input depend on the output, $\vec{x}[t]$.
\item We would like to change the matrix multiplying $\vec{x}[t]$ such that  $|\lambda_i|<1$, so let's see what happens when we let $u[t] = K\vec{x}[t]$, where $K \in \mathbb R^{1 \times n}$.
\item Using this input we have: 
\begin{align*}
\vec{x}[t+1] = A\vec{x}[t] + Bu[t] + \vec{\omega}[t]\\
= A\vec{x}[t] + BK\vec{x}[t] + \vec{\omega}[t]\\
= (A + BK)\vec{x}[t] + \vec{\omega}[t]
\end{align*}

\item Strategically choosing K allows us to have specific $\lambda$'s for A + BK (Good!).
\item This process is called coefficient matching. 
\end{itemize}
\end{frame}


\begin{frame}{Example}
\begin{itemize}
\item Suppose we are given a controllable system defined by: 
\begin{align*}
\vec{x}[t+1] = \begin{bmatrix} 2&-1 \\
0&1
\end{bmatrix} \vec{x}[t] + \begin{bmatrix} 2 \\ -1 \end{bmatrix}u[t]
\end{align*}
\item Is the system stable? \pause No! $\lambda = 2, 1$
\item What if we let \[ u[t] = \begin{bmatrix} f_1 & f_2 \end{bmatrix}\vec{x}[t]\]  Then we have:   
\begin{align*}
\vec{x}(t+1) = \begin{bmatrix} -2 & -1 \\
0&1
\end{bmatrix}\vec{x}[t] + \begin{bmatrix} 2 \\ -1 \end{bmatrix}\begin{bmatrix} f_1 & f_2 \end{bmatrix}\vec{x}[t] 
\end{align*}
\item Solve for the values of $f_1$ and $f_2$ such that $\lambda_1 = -0.25$ and $\lambda_2 = 0$ \pause
\item The answer is $f_1 = -1.50$ and $f_2 = 0.25$ \pause
\item Although the process is very messy hopefully you see why eigenvalue placement is very important to stabilize systems. What about bigger matrices?
\end{itemize}
\end{frame}

\begin{frame}{Controllable Canonical Form}
\begin{itemize}
\item Controllable Canonical Form (CCF) for any controllable system is a special form that allows us to simplify the process of eigenvalue placement. It takes on the following form:
\begin{columns}

\begin{column}{0.25\textwidth}
\begin{align*}
A^* = 
\begin{bmatrix}
0&1&0 &\hdots &0\\
0& 0 & 1& \hdots&0\\
\vdots&  \hdots &  \hdots& \hdots&0\\
0 & \hdots &\hdots & \hdots &1\\
\alpha_{0} & \alpha_{1}& \hdots & \hdots & \alpha_{n-1}
\end{bmatrix}
\end{align*}
\end{column}
\begin{column}{0.25\textwidth}
\begin{align*}
B^* = 
\begin{bmatrix}
0\\
0\\
\vdots\\
0\\
1
\end{bmatrix}
\end{align*}
\end{column}
\end{columns}\pause
\item The characteristic polynomial of $A^*$ is $\lambda_n - \sum\limits_{i = 0}^{n-1} \alpha_i\lambda^i$ = 0.
\item So how does it help with eigenvalue placement? \pause The last row of this matrix determines the eigenvalues of $A^*$ so modifying the last row will allow us to (easily) modify the eigenvalues.  
\end{itemize}
\end{frame}

\begin{frame}{How to convert to CCF}
\begin{itemize}
\item Let $A,B$ be the matrices in standard form and let $A^*,B^*$ be the matrices in CCF. \pause
\item Recall the matrix we used to check controllability?\\ \pause 
\[C = \begin{bmatrix}
B & AB & \hdots & A^{n-1}B
\end{bmatrix}\] \\  
\[ C^* = \begin{bmatrix}
B^*& A^*B^* & \hdots & A^{*n-1}B^*
\end{bmatrix}\]
\item We then have $T := C^*C^{-1}$,  such that $A^* = TAT^{-1}$ and $B^* = TB$.
\item Remember, all controllable matrices with single input can be transformed into CCF!
\end{itemize}
\end{frame}

\begin{frame}{Example}
Consider the following discrete time system:
\begin{align*}
\vec{x}[t+1] = \begin{bmatrix} 
-2&2&0\\
2&-4&2\\ 
0&1&-1
\end{bmatrix}\vec{x}[t] + \begin{bmatrix} 1 \\ 0 \\0 \end{bmatrix}u[t] 
\end{align*}

\begin{enumerate}[(a)]
\item Is the system stable? Is it controllable? 
\item Using an appropriate transformation ($\vec{z}[t] = T\vec{x}[t]$), bring the system to controllable canonical form. 
\item Using the state feedback $u[t] =$ \[\begin{bmatrix} f_1 & f_2 & f_3  \end{bmatrix}\] $\vec{z}[t]$ bring the eigenvalues of the system to $0,0.75, -0.25$.
\end{enumerate}
\end{frame}

\begin{frame}{Solutions to Example}
    \vspace{-2ex}
\begin{enumerate}[(a)]
\item The characteristic polynomial is: $\lambda^3 + 7\lambda^2 + 8\lambda = \lambda(\lambda^2 +7\lambda +8) = 0$, therefore the eigenvalues of A are $\{0,-5.56, -1.44\}$. As we can see there are $|\lambda_i|>1$ therefore the system is not stable. \\
The controllability matrix $C =$ \[\begin{bmatrix}
1&-2&8\\
0&2&-12\\ 
0&0&2
\end{bmatrix}\]
$C$ has full rank so the system is controllable

\item As we previously mentioned the coefficients of the characteristic polynomial are closely related to the last row of the $A^*$ matrix. Therefore, the CCF of the system is: 
\begin{align*}
\vec{z}[t+1] = 
\begin{bmatrix} 
0&1&0\\
0&0&1\\ 
0&-8&-7
\end{bmatrix}\vec{x}[t] + \begin{bmatrix} 0 \\ 0 \\1 \end{bmatrix}u[t] 
\end{align*}
\end{enumerate}
\end{frame}

\begin{frame}{Example Solutions Continued}
\begin{enumerate}[(c)]
\item Our system then becomes: 
\begin{align*}
\vec{z}[t+1] = 
\begin{bmatrix} 
0&1&0\\
0&0&1\\ 
f_1&f_2-8&f_3-7
\end{bmatrix}\vec{x}[t] 
\end{align*}
Which means its characteristic polynomial is : $\lambda^3 - (f_3 -7)\lambda^2 -(f_2 - 8)\lambda - f_1$ = 0. \\
Now, we know the characteristic polynomial should be $\lambda(\lambda - \frac{3}{4})(\lambda + \frac{1}{4})$, so we can equate the two and solve for the feedback vector $\vec{f}\hspace{0.1cm}^T = \begin{bmatrix}  0 & \frac{1}{2}& \frac{3}{16}\end{bmatrix}$.
\end{enumerate}
\end{frame}

