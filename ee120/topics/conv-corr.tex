\section{Convolution and Cross-Correlation}

\begin{frame}{Convolution}
    \begin{itemize}
        \item Convolution is a mathematical operation that takes in two signals, and outputs a third signal as the result of flipping one function and "shifting" it over the other. 

        \item Convolution is important because it allows us to determine the output of an LTI system given its provided input and its impulse response. 
            \[
                f[n]*g[n] = \sum_{k=-\infty}^{\infty} f[k]\cdot g[n-k]
            \]
            \[
                f(t)*g(t) = \int_{-\infty}^{\infty} f(\tau) \cdot g(t- \tau) d\tau
            \]
    \end{itemize}
\end{frame}
\begin{frame}{Useful Things to Know About Convolution}
    \begin{itemize}
        \item It is commutative in both DT and CT; $x*y = y*x$ 
        \item $x[n] * \alpha\,\delta[n-N] = \alpha\, x[n-N]$ (and likewise for CT)
        \item Convolution between two same-size rectangles gives a triangle
        \item Convolution between two different size rectangles gives a trapezoid
    \end{itemize}
\end{frame}
\begin{frame}{Graphical Interpretation of Convolution (CT)}
    \begin{itemize}
        \item In the formula, the function with the $t-\tau$ input is being flipped and shifted over the other one.
            \[
                f(t)*g(t) = \int_{-\infty}^{\infty} f(\tau) \cdot g(t- \tau) d\tau
            \]
        \item In CT, when drawing the signals being convolved, make sure to remember to graph them on the \textbf{tau} ($\tau$) axis, not on the time axis -- we are integrating with respect to $d\tau$ here! The output signal will be drawn on the standard time axis. 
        \item With DT, draw the operands on the k axis, and the output on the n axis. 
    \end{itemize}
\end{frame}
\begin{frame}{Convolution Practice}
    Determine the length of $x*y$ if: 
    \begin{itemize}
        \item They are DT, with $x[n]$ of length 4 and $y[n]$ of length 5
        \item They are CT, with $x(t)$ of length 4 and $y(t)$ of length 5
    \end{itemize}
\end{frame}
\begin{frame}{Convolution Practice}
    Determine the length of $x*y$ if: 
    \begin{itemize}
        \item They are DT, with $x[n]$ of length 4 and $y[n]$ of length 5
        \begin{itemize} 
            \item \textcolor{gray}{8. For DT signals, convolution of a length N and M signal results in a length N+M-1 signal.}
        \end{itemize}
        \item They are CT, with $x(t)$ of length 4 and $y(t)$ of length 5
        \begin{itemize} 
            \item \textcolor{gray}{9. For CT signals, convolution of a length N and M signal just results in a length N+M signal.}
        \end{itemize}
    \end{itemize}
\end{frame}
\begin{frame}{Convolution Practice}
    Convolve $x*y$ if: 
    \begin{itemize}
        \item $x[n] = u[n+2]-u[n-2]$ and $y[n]=u[n+1]-u[n-1]$.
        \item $x(t) = 1$ and $y[t] = 2(u(t+2)-u(t-2))$
    \end{itemize}
    \textcolor{red}{Try doing these with as little math as possible.}
\end{frame}
\begin{frame}{Convolution Practice}
    Convolve $x*y$ if:
    \begin{itemize}
        \item $x[n] = u[n+2]-u[n-2]$ and $y[n]=u[n+1]-u[n-1]$.
        \begin{itemize}
            \item \textcolor{gray}{$x[n]*y[n]=x[n]*(\delta[n+1]+\delta[n])=x[n+1]+x[n]$}
            \begin{figure}[H]
            \centering
            \begin{tikzpicture}
              \begin{groupplot}[group style={
                    group size=1 by 1
                  },
                  width=2in,
                  height=1.5in,
                  axis lines=middle,
                  xmin=-4.5,
                  xmax=3.5,
                  ymin=0,
                  ymax=2.5,
                  xlabel=$k$]
                \nextgroupplot[ylabel={$x*y=x[n+1]+x[n]$}, xtick={-3, -1, 1, 3},ytick={0,1,2}];
                \addplot+[ycomb, line width=1.5pt] plot coordinates {(-5, 0) (-4, 0) (-3, 1) (-2, 2) (-1, 2) (0, 2) (1, 1) (2, 0) (3, 0) (4, 0) (5, 0)};
                \end{groupplot}
            \end{tikzpicture}
          \end{figure}
        \end{itemize}
        \item $x(t) = 1$ and $y[t] = 2(u(t+2)-u(t-2))$
        \begin{itemize}
            \item \textcolor{gray}{This is convolving a horizontal line of height 1 with a rectangle of height 2 and width 4. The "overlap" is always going to have area 8 $\Rightarrow x(t)*y(t)=8$}
        \end{itemize}
    \end{itemize}
\end{frame}
\begin{frame}{Cross-Correlation}
    \begin{itemize}
        \item Cross-correlation is a similar operation to convolution, that also involved a shift and sum of two signals. 
        \item It's not \textit{exactly} convolution! 
        \begin{itemize}
            \item Not commutative
            \item Signal is not flipped before shifting! Is directly shifted. 
        \end{itemize}
            \[
                R_{f,g}[n] = \sum_{k=-\infty}^{\infty} f[k]\cdot g[n+k]
            \]
            \[
                R_{f,g}(t) = \int_{-\infty}^{\infty} f(\tau) \cdot g(t+ \tau) d\tau
            \]
    \end{itemize}
\end{frame}
\begin{frame}{Cross-Correlation Practice}
    Prove that cross-correlation is not commutative. More specifically, prove that $R_{x,y}[n] = R_{y,x}[-n]$, where $$R_{x,y}[n] = \sum_{k=-\infty}^{\infty} x[k]\cdot y[n+k].$$
\end{frame}
\begin{frame}{Cross-Correlation Practice}
    Prove that cross-correlation is not commutative. More specifically, prove that $R_{x,y}[n] = R_{y,x}[-n]$, where $$R_{x,y}[n] = \sum_{k=-\infty}^{\infty} x[k]\cdot y[n+k].$$
    \textcolor{gray}{R_{x,y}[n] = \sum_{k=-\infty}^{\infty} x[k]\cdot y[n+k] = \sum_{w=-\infty}^{\infty} x[w-n]\cdot y[w] = \sum_{w=-\infty}^{\infty} y[w] \cdot x[w-n] = R_{y,x}[-n]}
\end{frame}
\end{document}