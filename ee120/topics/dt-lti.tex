\section{DT LTI Systems}

\begin{frame}{Discrete Signals and Systems}

    \begin{itemize}
        \item A discrete signal is described by a function
            \[
                x: \mathbb{Z} \rightarrow \mathbb{R}
            \]
        \item A delta function is the unit impulse and any signal can be decomposed into a sum of deltas
            \[
                \delta[n] =
                \begin{cases}
                    1 & n= 0 \\
                    0 & n\ne 0
                \end{cases}
            \]
            \[
                x[n] = \sum_{k=-\infty}^{\infty} x[k]\delta[n-k] \forall x
            \]
        \item A discrete system, $H$, takes an input signal, $x$, and yields an output signal, $y$
    \end{itemize}
\end{frame}

\begin{frame}{LTI Systems and Impulse Response}
\begin{itemize}
    \item The impulse response is the output of a system for a $\delta[n]$ input
    \[
        x[n] = \delta[n] \overset{H}{\longrightarrow} y[n] = h[n]
    \]
    \item Linear and time-invariant systems are important because knowing its impulse response allows you to determine the response to any other input.
\end{itemize}

\vspace{20px}

\textcolor{red}{Verify that knowing $h[n]$ allows you to determine the output $y[n]$ for any input in an LTI system. \textit{Hint: decompose $x[n]$}}

\end{frame}

\begin{frame}{Proof and the Convolutional Sum}
    \underline{Proof}
    Decompose $x[n]$
    \[x[n] = \sum_k x[k]\delta[n-k]\]
    Recall the definition of the impulse response
    \[\delta[n] \overset{H}{\longrightarrow} h[n]\]
    Apply linearity and time-invariance to the definition of the impulse response
    \[\sum_k x[k]\delta[n-k] \overset{H}{\longrightarrow} \sum_k x[k]h[n-k]\]
    \underline{Convolution}
\begin{itemize}
    \item This is the convolutional sum, and denoted by $(x * h)[n]$
    \item Note: $(x * h)[n] = (h * x)[n]$
\end{itemize}
\end{frame}

\begin{frame}{LTI Example Problems}
    \textcolor{red}{Determine if the system described by the input-output relationship is 1) linear and 2) time-invariant. {\footnotesize\textit{Credit: Discussion \#1 EE120 Sp21}}}
    \begin{enumerate}
        \item $y[n] = 7x[n+1]$
        \item $y[n] = x[n]x[n-1]$
        \item $y[n] = e^{x[n]}$
        \item $y[n] = x[-n]$
        \item $y[n] = v[n]x[n]$, where $v$ is some fixed signal
    \end{enumerate}
\end{frame}

\begin{frame}{LTI Example Solutions}
    \begin{enumerate}
        \item $y[n] = 7x[n+1]$
        \begin{itemize}
            \item Linear: \textcolor{red}{\textbf{Yes}}
            \item Time-invariant: \textcolor{red}{\textbf{Yes}}
        \end{itemize}
        \item $y[n] = x[n]x[n-1]$
        \begin{itemize}
            \item Linear: \textcolor{red}{\textbf{No}, any scaling factor becomes squared at the output}
            \item Time-invariant: \textcolor{red}{\textbf{Yes}}
        \end{itemize}
        \item $y[n] = e^{x[n]}$
        \begin{itemize}
            \item Linear: \textcolor{red}{\textbf{No}, exponentials are non-linear (verify at $n=0$)}
            \item Time-invariant: \textcolor{red}{\textbf{Yes}}
        \end{itemize}
        \item $y[n] = x[-n]$
        \begin{itemize}
            \item Linear: \textcolor{red}{\textbf{Yes}}
            \item Time-invariant: \textcolor{red}{\textbf{No}, $\delta[n-1]$ input results in $\delta[-n-1]$}
        \end{itemize}
        \item $y[n] = v[n]x[n]$, where $v$ is some fixed signal
        \begin{itemize}
            \item Linear: \textcolor{red}{\textbf{Yes}}
            \item Time-invariant: \textcolor{red}{\textbf{No}, consider when $v[n] = \delta[n]$ for $x[n]$ and $\Tilde{x}[n] = x[n-1]$; $y[0] = x[0]$ and $\Tilde{y[0]} = x[-1]$}
        \end{itemize}
    \end{enumerate}
\end{frame}

\begin{frame}{DT System Example Problem}
    \textcolor{red}{Consider the \textit{linear} DT system H}
    \[
    \delta[n-k] \overset{H}{\longrightarrow} h_k[n] = \alpha^{|k|}u[n-k] \quad \forall |\alpha| < 1
    \]

    \vspace{30px}

    \textcolor{red}{Show that the output, $y[n]$, to a general input, $x[n]$ is}
    \[
        y[n] = \sum_{k=-\infty}^{+\infty}x[k]h_k[n]
    \]
    {\footnotesize\textcolor{red}{\textit{Credit: Midterm \#1 EE120 Sp20}}}

\end{frame}

\begin{frame}{DT System Example Solution}
    \textcolor{red}{Decompose $x[n]$}
    \[x[n] = \sum_{m=-\infty}^{+\infty}x[m]\delta[n-m]\]
    \textcolor{red}{Recall the definition of the system $H$}
    \[\delta[n-m] \overset{H}{\longrightarrow} h_m[n]\]
    \textcolor{red}{Apply linearity}
    \[x[n] = \sum_{m=-\infty}^{+\infty}x[m]\delta[n-m] \overset{H}{\longrightarrow} \sum_{m=-\infty}^{+\infty}x[m]h_m[n] = y[n]\]

    Notice the differences in this sum from the usual convolutional sum.
\end{frame}

\begin{frame}{DT System Example Problem Continued}
    \textcolor{red}{Derive a closed-form expression (no summations) for the output $y[n]$ when the input $x[n] = u[n]$, the unit step}

    Recall:
    \[h_k[n] = \alpha^{|k|}u[n-k] \quad \forall |\alpha| < 1\]
    \[y[n] = \sum_{k=-\infty}^{+\infty}x[k]h_k[n]\]
    \[\sum_{k=M}^N \alpha^k =
        \begin{cases}
            \frac{\alpha^{N+1}-\alpha^M}{\alpha-1} & \alpha\ne 1 \\
            N-M+1 & \alpha= 1
        \end{cases}
    \]
\end{frame}

\begin{frame}{DT System Example Solution}
    \[y[n] = \sum_{k=-\infty}^{+\infty}x[k]h_k[n]\]
    \textcolor{red}{Change the bounds and plug in for $x[k]$ and $h_k[n]$}
    \[y[n] = \sum_{k=0}^{+\infty}1\cdot \alpha^{|k|}u[n-k]\]
    \[y[n] = u[n] + \alpha u[n-1] + \alpha^2 u[n-2] + \dots\]
    \[y[0] = 1\]
    \[y[1] = 1+\alpha\]
    \[y[2] = 1+\alpha+\alpha^2\]
    \[y[n] = \sum_{p=0}^n \alpha^p = \frac{\alpha^{n+1}-1}{\alpha -1}\]

\end{frame}
